\thispagestyle{empty}
\newcommand{\TitleCHS}{ Nvidia新架构GPU为机器学习应用带来的性能提升的研究与评估} %中文标题

\newcommand{\TitleENG}{Research on performance of ML applications using Nvidia new GPUs} %英文标题

\newcommand{\Author}{刘子汉} %作者名字

\newcommand{\StudentID}{10152130243} %学号

\newcommand{\Department}{计算机科学与软件工程学院} %学院

\newcommand{\Major}{计算机科学与技术} %专业

\newcommand{\Supervisor}{钱莹} %导师名字

\newcommand{\AcademicTitle}{副教授} %导师职称

\newcommand{\CompleteYear}{2019} %毕业年份

\newcommand{\CompleteMonth}{5} %毕业月份

\newcommand{\KeywordsCHS}{Tensor Core,TensorRT,通用矩阵乘法,图灵架构 } %中文关键词

\newcommand{\KeywordsENG}{Tensor Core, TensorRT, GEMM, Turing Architecture} %英文关键词

\centerline{\bfseries \sffamily \zihao{-3}\TitleCHS}
\renewcommand\abstractname{\sffamily\zihao{-4} 摘要}
\begin{abstract}\zihao{5}\rmfamily
	\par 本文主要针对Nvidia新架构的GPU(图灵架构)为机器学习应用带来的性能提升进行研究,由于目前实际使用中的应用很难达到Nvidia官方宣传的性能提升幅度,故本文将从问题类型、代码结构结合硬件、指令特征对这一现象进行研究,并提出相应的建议。本文主要采用定量方法,通过不同世代的硬件和SDK进行横向比较,以及同一世代硬件、SDK和不同类型应用进行纵向比较;并总结出特征。在研究中较为重要的部分为新硬件中加入的张量核心(Tensor Core)以及对应的线性代数库CUTLASS,文章将通过混合矩阵运算、矩阵乘法、卷积运算等对其进行评估;其他还涉及了传统的矩阵运算库CUBLAS、模型优化器TensorRT以及最为基本的浮点计算、内存种类等。
	\par 根据实验结果,新架构硬件中张量核心对于机器学习应用的类型、计算类型、超参数等条件敏感;要达到期望的性能,输入数据规模、形状、运算占比等方面有较为严苛的需求;在矩阵较为稀疏、输入规模较小时CUSPARSE稀疏矩阵库和基于纹理内存的方法能取得更高性能;而计算输入较为规律、符合硬件形状时张量核心能带来显著提升。至于网络推理阶段,TensorRT在各种情况下均能带来明显的提升。在实际应用中,训练阶段应根据任务特征合理选择硬件、SDK和内存系统使用;而在推理阶段应利用Tensor Core提升吞吐量。
	\newline
	\newline
	{\bfseries \sffamily\zihao{5} 关键词:} \zihao{5}{\rmfamily \KeywordsCHS}
\end{abstract}