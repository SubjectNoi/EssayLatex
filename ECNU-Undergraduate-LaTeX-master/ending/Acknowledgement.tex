\clearpage                              %
\section*{致谢}                         %通常你不需要修改这部分内容
\addcontentsline{toc}{section}{致谢}    %
\par 距离2015年09月,已经过去了将近四年。
\par 四年前刚刚踏入大学校园的种种仿佛还就在眼前。回忆起曾经高中学业的紧张、择校选专业时的忐忑,再看现在未来已经基本定型的情况,不由心生感慨:能找到自己真正的兴趣并在目前的实习、之后的进一步学习和工作中予以实践,着实是一件幸运的事。
\par 首先,感谢家人的陪伴,感谢父母对于我学业无论是在经济上还是在生活上全力的支持,没有你们的帮助,我甚至没有能力承担学业。有时我会学习到深夜,在大家都已经进入梦乡的时间,你们还会为我煮上一碗面。正是因为你们,我才能完成论文、完成学业。
\par 在大学的前几年,我也迷茫过,在别的同学做项目、打比赛的时候我也焦虑过;但是幸好我被给予了一个延续我高中以来的梦想:研究硬件、系统结构的机会,钱老师的并行计算课程、魏老师的计算机组成课程、王老师的嵌入式原理课程、肖老师的算法课程、陆老师的C++课程、等等……这些课程可谓是为我打开了一扇大门,我得以系统地学习我曾感兴趣的知识;也正是因为这门课程,我也确定了本文的主题,确定了研究生的学习方向,确定了目前的实习,以及将来的职业目标。
\par 说到职业,这里不得不提到在英伟达(NVIDIA)实习时,公司以及同事对我的帮助,正是因为这份实习,让我有机会接触到无数的涉及GPU底层架构的文档,让我有机会深入到GPU级别的汇编代码进行编程,这些经验、资料对本文的写作带来了极大的帮助,当然,这一切都归功于Edward先生愿意给予我这次实习机会,并悉心指导我。
\par 在论文的撰写中,我还得到了许多同学的协助;有同样对并行计算感兴趣的吕同学与我耐心的探讨,吕同学同时也熟悉\LaTeX 的使用,在完善论文格式时,我也得到了他的许多帮助;有姚同学给我提出的实验方面建议;有沈同学与我分享行业最新信息、探讨最新硬件;还有各位一起娱乐的群友们为我带来的欢乐与放松……这些无一不让我在紧张的论文撰写中得以卸下一些压力。当然,不只是大学中的同学们,这里也感谢我自初中以来的同学,也是我的女友的Vega姜小姐九年以来的陪伴以及在身心上给予我的支持。
\par 论文总有一天会完成上交,学生生涯总有一天会迎来结束。然而对新知识的探求正是支撑起我们计算机学子前进的基石。不求对世界做出什么改变,不求对人类做出什么贡献,只求在未来的道路里不忘初心、坚守道德、尽力而为、劳逸结合、保持童心、有始有终、乐观对待、做自己想做的事,并且无憾一生。
\par Arrivederci.
