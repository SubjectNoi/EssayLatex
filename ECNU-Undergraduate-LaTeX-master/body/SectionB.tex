\newpage
\section{背景及相关工作}
\setcounter{table}{0}
\setcounter{figure}{0}
\subsection{NVIDIA GPU硬件结构}
\subsubsection{GPU芯片总体结构} 
\par 在介绍新老架构区别之前,本节首先自顶向下简要介绍一下NVIDIA GPU芯片的结构。一块GPU芯片拥有若干图形处理器簇(Graphics Processing Cluster, GPC),由外围总线进行调度管理;一个图形处理器簇上有若干纹理处理器簇(Texture Processing Cluster, TPC);需要注意的是以上两种结构在编写CUDA程序时并不暴露。一个纹理处理器粗上有若干流多处理器单元(Stream Multiprocessor, SM),也是本文关注的重点。流多处理器单元被一个线程块调度器管理,所有流多处理器单元通过全局内存总线经过L2缓存共享全局内存。每个流多处理器单元中由若干流处理器(Stream Processpr, SP),然而这一概念随着流多处理器单元中运算单元种类的增加而被弱化了。在一个流多处理器单元内部的流处理器共享一个指令缓存,每个流处理器拥有自己的线程束调度器与寄存器文件;流处理器中包含若干种执行单元,有浮点单元,整数单元,在新架构中还加入了张量单元(Tensor Core),在RTX 2080TI上具体的参数为:一个SM包含64个单精度浮点算术单元,32个双精度浮点算术单元,64个32位整形算术单元,8个混合精度张量单元,4个线程束调度器和16个特殊功能单元;所有流处理器通过显存纵横矩阵(CrossBar)访问共享内存,或被称为L1缓存\parencite{EXPLORING}。
\subsubsection{流多处理器单元(SM)} 
\par 上文提到过,六多处理器单元(SM)是本文关注的重点,其原因是每一次NVIDIA GPU芯片更新都会伴随着其计算能力(Compute Capability)的更新,计算能力指的是流多处理器单元(SM)支持的运算的等级,分为Major和Minor。其中Major代号代表架构的更新,这也会带来许多新的硬件支持的运算,而Minor代号则代表同一架构下不同定位的流多处理器产品。如伏特架构的计算能力为7.2,图灵架构的计算能力为7.5,Major代号一样就代表这两种架构其实并无太大修改,而Minor代号则代表伏特架构中流多处理器的类型是Heavy,图灵架构中流多处理器的类型是Lite。Lite和Heavy一般用于区分消费级/工作站级GPU,分别对应GeForce和Tesla代号。
\subsubsection{存储模型与管理} 
\par NVIDIA GPU的存储模型与其存储管理系统也是另一个重点。传统CPU编程模型中,寄存器、缓存等资源都是由CPU自行管理,而不开放给程序员。其原因在于CPU拥有的寄存器、缓存资源较为紧缺,为提高指令级并行能力,需要采用多队列乱序发射与寄存器重命名等技术。相对得,GPU有较为充足的物理寄存器、缓存资源,程序员也对这部分资源掌握有一定的控制权\parencite{CUDAPROG}。CUDA中的存储设备如表\ref{table-存储}所示。
\begin{table}
	\centering
	\renewcommand{\thetable}{\arabic{section}-\arabic{table} }
	\renewcommand{\tablename}{表}
	\caption{CUDA存储系统层级}
	\addtocounter{table}{-1}
	\renewcommand{\thetable}{\arabic{section}-\arabic{table} }
	\renewcommand{\tablename}{Table}
	\caption{CUDA storage system hierarchy}
	\begin{tabular}{cccc}
		\toprule
		项目				&	大小			&	延迟(时钟周期)	&	访问权限	\\
		\midrule
		寄存器文件		&	8KB-64KB/SM		&	$ 10^0 $	& GPU端	\\
		共享内存(L1,L2)	&	16KB-128KB/SM	&	$ 10^1 $	&	GPU端\\
		常量内存		&	N/A				&	N/A		&	N/A	\\
		纹理内存		&	N/A				&	N/A		&	N/A \\
		全局内存		&	-GB				&	$ 10^2 $	&	CPU端/GPU端 \\
		\bottomrule
	\end{tabular} \label{table-存储}
\end{table}
\par 需要注意的是,常量内存与纹理内存都是全局内存的一种虚拟地址形式。和常量内存一样,纹理内存也是一种只读内存;但是在访存、缓存加载方式上与其余存储系统存在较大差异,而这种差异会在某些应用中极大提高性能,在本文的实验中大量利用了纹理内存的特性,故将在下一节详细介绍纹理内存。
\subsubsection{纹理内存(Texture Memory)}
\par 
\subsection{伏特/图灵架构新硬件}     
\subsubsection{在流多处理器单元层面的差异}
\subsubsection{张量核心(Tensor Core)}                                              
\subsection{软件}
\par 本节将自顶向下介绍NVIDIA GPU硬件对应的不同层级的编程软件。
\subsubsection{机器学习框架(Tensor Flow)}
\subsubsection{CUDA C}
\subsubsection{机器码与中间代码(SASS, PTX)}
\subsection{基于GPU的机器学习应用}