\clearpage                                          %       


\setcounter{page}{1} 
\pagenumbering{Roman}% Roman page numbers                             %
\newcommand{\TitleCHS}{ Nvidia新架构GPU为机器学习应用带来的性能提升的研究与评估} %中文标题

\newcommand{\TitleENG}{Research on performance of ML applications using Nvidia new GPUs} %英文标题

\newcommand{\Author}{刘子汉} %作者名字

\newcommand{\StudentID}{10152130243} %学号

\newcommand{\Department}{计算机科学与软件工程学院} %学院

\newcommand{\Major}{计算机科学与技术} %专业

\newcommand{\Supervisor}{钱莹} %导师名字

\newcommand{\AcademicTitle}{副教授} %导师职称

\newcommand{\CompleteYear}{2019} %毕业年份

\newcommand{\CompleteMonth}{5} %毕业月份

\newcommand{\KeywordsCHS}{Tensor Core,TensorRT,通用矩阵乘法,图灵架构 } %中文关键词

\newcommand{\KeywordsENG}{Tensor Core, TensorRT, GEMM, Turing Architecture} %英文关键词
                          %
\renewcommand\abstractname{\bfseries \heiti \zihao{3}摘\hspace{1cm}要}  %
\addcontentsline{toc}{section}{摘要}
\begin{abstract}\zihao{5}\songti                    %
	%%%%%%%%%%%%%%%%%%%%%%%%%%%%%%%%%%%%%%%%%%%%%%%%%%%%%
	\vspace{3em}
	\par 本文主要针对Nvidia新架构的GPU(图灵架构)为机器学习应用带来的性能提升进行研究,由于目前实际使用中的应用很难达到Nvidia官方宣传的性能提升幅度,故本文将从问题类型、代码结构结合硬件、指令特征对这一现象进行研究,并提出相应的建议。本文主要采用定量方法,通过不同世代的硬件和SDK进行横向比较,以及同一世代硬件、SDK和不同类型应用进行纵向比较;并总结出特征。在研究中较为重要的部分为新硬件中加入的张量核心(Tensor Core)以及对应的线性代数库CUTLASS,文章将通过混合矩阵运算、矩阵乘法、卷积运算等对其进行评估;其他还涉及了传统的矩阵运算库CUBLAS、模型优化器TensorRT以及最为基本的浮点计算、内存种类等。
	\par 根据实验结果,新架构硬件中张量核心对于机器学习应用的类型、计算类型、超参数等条件敏感;要达到期望的性能,输入数据规模、形状、运算占比等方面有较为严苛的需求;在矩阵较为稀疏、输入规模较小时CUSPARSE稀疏矩阵库和基于纹理内存的方法能取得更高性能;而计算输入较为规律、符合硬件形状时张量核心能带来显著提升。至于网络推理阶段,TensorRT在各种情况下均能带来明显的提升。在实际应用中,训练阶段应根据任务特征合理选择硬件、SDK和内存系统使用;而在推理阶段应利用Tensor Core提升吞吐量。
	\newline
	\newline
	{\bfseries \sffamily\zihao{5} 关键词:} \zihao{5}{\rmfamily \KeywordsCHS}
\end{abstract}                                                    %
%通常你不需要修改这部分内容
\clearpage                         
\renewcommand\abstractname{\bfseries \zihao{3}Abstract}  
\addcontentsline{toc}{section}{Abstract}                 %
\begin{abstract}\zihao{5}                                         %
	%%%%%%%%%%%%%%%%%%%%%%%%%%%%%%%%%%%%%%%%%%%%%%%%%%%%%%%%%%%%%%%%%%%
	\vspace{3em}
	\par This paper is focusing on the performance improvement in Machine Learning application brougnt by Nvidia’s new architecture (Turing architecture) GPU. Since currently the Machine Learning application actually in used can hardly get as much improvement as mentioned in Nvidia’s official White Paper, so, this paper will research this situation through the type of the application, the structure of the source code combining with feature of the hardware and instructions, thus give corresponding recommendation about coding. This paper uses quantitative methods, doing both horizontal comparation with hardware and SDK of different generations and vertical comparation with different types of problem running on the same generation of hardware and SDK, through which the pattern and feature can be extracted. Among all the new features, the most important is Tensor Core and corresponding library CUTLASS (CUDA Template Linear Algebra Subroutine), this paper evaluate this unit through GEMM, Matrix Multiple, Convolution, etc. Also, traditional matrix library CUBLAS, optimizer TensorRT, Float Point and GRAM are also mentioned.  
	\par In the conclusion, Tensor Core in the new architecture GPU is very sensitive to the type of applications, type of calculations, meta parameter, etc., to achieve expected performance, the scale of the data, shape of the data and type of calculations should be well fit to the hardware. Moreover, in some situation including the input matrixs are sparse and the scale of the input data is small, library oriented to sparse matrix (CUSPARSE) and methods based on texture memory will gain much higher performance, and in situation that the input fit the hardware well, the Tensor Core can bring the application a significant improvement in performance. When it comes to the inference stage, TensorRT can bring a significant improvement in almost all the situation. 
	\par So, in the training stage of actual application, the usage of hardware, SDK, memory, etc. should be chosen appropriate based on the feature of the applications, and in the inference stage, do not hesitate to use TensorRT!
	\newline
	\newline
	{\bfseries \zihao{5} Keywords:} {\zihao{5} \KeywordsENG}        %   
\end{abstract}                                                 %
%%%%%%%%%%%%%%%%%%%%%%%%%%%%%%%%%%%%%%%%%%%%%%%%%%%%%%%%%%%%%%%%