\thispagestyle{empty}
%\centerline{\bfseries \zihao{-3}\TitleENG}
\renewcommand\abstractname{\zihao{-3} Abstract}
\begin{abstract}\zihao{5}
    \par This paper is focusing on the performance improvement in Machine Learning application brougnt by Nvidia’s new architecture (Turing architecture) GPU. Since currently the Machine Learning application actually in used can hardly get as much improvement as mentioned in Nvidia’s official White Paper, so, this paper will research this situation through the type of the application, the structure of the source code combining with feature of the hardware and instructions, thus give corresponding recommendation about coding. This paper uses quantitative methods, doing both horizontal comparation with hardware and SDK of different generations and vertical comparation with different types of problem running on the same generation of hardware and SDK, through which the pattern and feature can be extracted. Among all the new features, the most important is Tensor Core and corresponding library CUTLASS (CUDA Template Linear Algebra Subroutine), this paper evaluate this unit through GEMM, Matrix Multiple, Convolution, etc. Also, traditional matrix library CUBLAS, optimizer TensorRT, Float Point and GRAM are also mentioned.  
    \par In the conclusion, Tensor Core in the new architecture GPU is very sensitive to the type of applications, type of calculations, meta parameter, etc., to achieve expected performance, the scale of the data, shape of the data and type of calculations should be well fit to the hardware. Moreover, in some situation including the input matrixs are sparse and the scale of the input data is small, library oriented to sparse matrix (CUSPARSE) and methods based on texture memory will gain much higher performance, and in situation that the input fit the hardware well, the Tensor Core can bring the application a significant improvement in performance. When it comes to the inference stage, TensorRT can bring a significant improvement in almost all the situation. 
    \par So, in the training stage of actual application, the usage of hardware, SDK, memory, etc. should be chosen appropriate based on the feature of the applications, and in the inference stage, do not hesitate to use TensorRT!
    \newline
    \newline
    {\bfseries \zihao{5} Keywords:} {\zihao{5} \KeywordsENG}
\end{abstract}