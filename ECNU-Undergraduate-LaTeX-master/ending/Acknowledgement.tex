\clearpage                              %
\section*{致谢}                         %通常你不需要修改这部分内容
\addcontentsline{toc}{section}{致谢}    %
\par 距离2015年09月,已经过去了将近四年。
\par 四年前刚刚踏入大学校园的种种仿佛还就在眼前。回忆起曾经高中学业的紧张、择校选专业时的忐忑,再看现在未来已经基本定型的情况,不由心生感慨:能找到自己真正的兴趣并在目前的实习、之后的进一步学习和工作中予以实践,着实是一件幸运的事。
\par 首先,感谢家人的陪伴,感谢父母对于我学业无论是在经济上还是在生活上全力的支持,没有你们的帮助,我甚至没有能力承担学业。有时我会学习到深夜,在大家都已经进入梦乡的时间,你们还会为我煮上一碗面。正是因为你们,我才能完成论文、完成学业。
\par 在大学的前几年,我也迷茫过,在别的同学做项目、打比赛的时候我也焦虑过;但是幸好我被给予了一个延续我高中以来的梦想:研究硬件、系统结构的机会,钱老师的并行计算课程、魏老师的计算机组成课程、王老师的嵌入式原理课程、肖老师的算法课程、陆老师的C++课程、等等……这些课程可谓是为我打开了一扇大门,我得以系统地学习我曾感兴趣的知识;也正是因为这门课程,我也确定了本文的主题,确定了研究生的学习方向,确定了目前的实习,以及将来的职业目标。
\par 说到职业,这里不得不提到SAP提供了我第一份实习工作。俗话说万事开头难,非常感谢James对我的认可,以及Eric, Radar, Reno对我的悉心指导,同时还感谢机缘巧合同样在SAP工作的学校的前辈July。在SAP之后,我进入了英伟达(NVIDIA)工作,正是因为这份工作,让我有机会接触到无数涉及GPU底层架构、指令的文档,让我有机会深入SASS级别的汇编代码进行编程。这些经验、资料极大的辅助了本文的写作。当然,这还归功于Edward先生给予我的机会以及Roc, Jiacheng, Colin耐心的指导。
\par 在论文的撰写中,我得到了许多同学的协助;有同样对并行计算感兴趣的英哲与我耐心的探讨,英哲同时也熟悉\LaTeX 的使用,在完善论文格式时,我也得到了他的许多帮助;有舜禹给我提出的实验方面建议;有佳易与我分享行业最新信息、探讨最新硬件;有一起娱乐的群友Edstrictland, FancyNiya, sarahcc, Niwako, ParrotQ, DestinyCoder, sylviaaaaaa, 松柏|梧桐等,他们或是与我一同攻略但丁必死难度,或是一同传承初火、狩猎暗灵,或是一同拯救变若卿子、斩断龙胤,或是一同在未知的空域咬尾、缠斗;还有协助我管理群的徐小姐、粥小姐、刘小姐、杨部长、黄先生……这些无一不让我在紧张的论文撰写中得以卸下一些压力。当然,不只是大学中的同学们,这里也感谢我自初中以来的同学,也是我的女友的Vega姜小姐九年以来的陪伴以及在身心上给予我的支持。
\par 这里也特别感谢Mr.Quin, Ywwuyi, 全撸剩饭, 那须桃子, Quitzera, Maaya Uchida, Ai Kayano, Miku Nakano, Nino Nakano, Mai Sakurajima, Nanami Arihara等人的作品为我带来的欢乐时光,也希望有生之年,能够看到秦先生的黑暗剑完结。
\par 论文总有一天会完成上交,学生生涯总有一天会迎来结束。然而对新知识的探求正是支撑起我们计算机学子前进的基石。不求对世界做出什么改变,不求对人类做出什么贡献,只求在未来的道路里不忘初心、坚守道德、尽力而为、劳逸结合、保持童心、有始有终、乐观对待、做自己想做的事,并且无憾一生。
\par Arrivederci.
